\documentclass[a4paper,10pt]{article}

\usepackage[margin=0.5in]{geometry}
\usepackage[pdftex]{graphicx}

\usepackage{algorithm}
\usepackage{algpseudocode}
\usepackage[slovak]{babel}
\usepackage[utf8x]{inputenc}


\DeclareTextFontCommand{\emph}{\bfseries}
\begin{document}

\begin{titlepage}
\begin{center}
\textsc{ }\\[2cm]
\textsc{\LARGE Masaryková univerzita}\\[0.2cm]
\textsc{\LARGE Fakulta informatiky}\\[0.5cm]
% Upper part of the page. The '~' is needed because \\
% only works if a paragraph has started.
\includegraphics[width=0.3\textwidth]{./filogo}~\\[3cm]


% Title
{ \huge \bfseries Teoretické základy informatiky a matematika \\[0.4cm] }
\textsc{\Large Vypracované otázky pre bakalárske SZZ}\\[0.5cm]

\vspace*{\fill}



% Bottom of the page
{\large Jar 2015}

\end{center}
\end{titlepage}

%\domsecttoc
\tableofcontents
\pagebreak

\section{Množiny, relácie, zobrazenia, čísla.} 
\textit{Základné množinové operácie, množinový kalkul, potenčná množina, kartézsky súčin. Relácie a ich vlastnosti - ekvivalencia a rozklady, usporiadanie a usporiadané množiny. Skladanie relácií, zobrazenia (injekcia, surjekcia, bijekcia). Elementárna teória čísiel (delitelnosť, Euklidov algoritmus, modulárne operácie).}
	\subsection{Množiny}
	Množina je súbor prvkov a je svojimi prvkami plne určená. Množina môže byť prvkom inej množiny. Množina nemusí mať konečný počet prvkov.
	
	Majme $ M = \{a, b\} = \{b, a\} = \{a, b, a\} $ a $ N = \{\{a\}, \{b, c, d, e\}\} $. Pre prvky množín $M$ a $N$ potom platí: $ a \in M $, $a \not\in N$, $\{a\} \in N$.
	
	Pokiaľ je množina prázdna (t.j. neobsahuje žiadne prvky - takúto množinu značíme $\emptyset$), potom platí $\emptyset \in \{\emptyset\}$ ale $\emptyset \not\in \emptyset$.
	
		\subsubsection{Základné pojmy}
		\emph{Mohutnosť množiny} je určená počtom jej prvkov. Mohutnosť množiny $A$ zapisujeme ako $|A|$.
		$$|\emptyset| = 0$$
		$$|\{\emptyset\}| = 1$$
		$$|\{a,b,c\}| = 3$$
		$$|\{\{a, b\}, c\}| = 2$$
	
	Množina je $A$ \emph{podmnožinou} množiny $B$ práve vtedy, keď každý prvok $A$ je prvkom $B$. Píšeme $A \subseteq B$. Tento vzťah nazývame tiež \emph{inklúzia}\footnote{Pre ľubovolnú množinu $S$ je relácia inklúzie čiastočným usporiadaním na množine $2^{A}$ (viď.~\ref{usporiadanie} a~\ref{potencna_mnozina})}. Zároveň vtedy platí, že $B$ je \emph{nadmnožinou} množiny $A$ (píšeme $B \supseteq A$).
	$$A \subseteq B \Leftrightarrow \forall x(x \in A \Rightarrow x \in B)$$
	
	Množiny si sú \emph{rovné} práve vtedy, keď $A \subseteq B$ a $B \subseteq A$.
	$$A = B \Leftrightarrow A \subseteq B \land B \subseteq A$$
	
	Množina $A$ je \emph{vlastnou podmnožinou} množiny $B$ práve vtedy, keď $A$ je podmnožinou $B$ a $A \not= B$.
	$$A \subset B \Leftrightarrow A \subseteq B \land \exists x \in B(x \not\in A)$$
	\subsection{Základné množinové operácie}
	Základnými množinovými operáciami nad množinami $A$ a $B$ sú:
	\begin{itemize}
		\item \emph{zjednotenie} - $A \cup B = \{x \mid x \in A \lor x \in B \}$
		\item \emph{prienik} - $A \cap B = \{x \mid x \in A \land x \in B \}$
		\item \emph{rozdiel} - $A \setminus B = \{x \mid x \in A \land x \not\in B \}$
		\item \emph{symetrický rozdiel} - $A \bigtriangleup B = (A \setminus B) \cup (B \setminus A)$
		\item \emph{doplnok} (komplement) - nech $A \subseteq M$. Doplnok množiny $A$ vzhľadom k množine $M$ je množina $\overline{A} = M \setminus A$ 
	\end{itemize}
	\subsection{Potenčná množina}\label{potencna_mnozina}
	\subsection{Usporiadanie}\label{usporiadanie}
\end{document}