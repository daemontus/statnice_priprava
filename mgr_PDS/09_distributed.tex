%%%%%%%%%%%%%%%%%%%%%%%%%%%%%%%%%%%%%%%%%
% Short Sectioned Assignment
% LaTeX Template
% Version 1.0 (5/5/12)
%
% This template has been downloaded from:
% http://www.LaTeXTemplates.com
%
% Original author:
% Frits Wenneker (http://www.howtotex.com)
%
% License:
% CC BY-NC-SA 3.0 (http://creativecommons.org/licenses/by-nc-sa/3.0/)
%
%%%%%%%%%%%%%%%%%%%%%%%%%%%%%%%%%%%%%%%%%

%----------------------------------------------------------------------------------------
%	PACKAGES AND OTHER DOCUMENT CONFIGURATIONS
%----------------------------------------------------------------------------------------

\documentclass[paper=a4, fontsize=11pt]{scrartcl} % A4 paper and 11pt font size

%\usepackage[T1]{fontenc} % Use 8-bit encoding that has 256 glyphs
%\usepackage{fourier} % Use the Adobe Utopia font for the document - comment this line to return to the LaTeX default
\usepackage[utf8]{inputenc} 
\usepackage[czech]{babel} % English language/hyphenation
\usepackage{amsmath,amsfonts,amsthm} % Math packages
\usepackage{paracol}

\usepackage{lipsum} % Used for inserting dummy 'Lorem ipsum' text into the template

\usepackage{sectsty} % Allows customizing section commands
\allsectionsfont{\centering \normalfont\scshape} % Make all sections centered, the default font and small caps

\usepackage{fancyhdr} % Custom headers and footers
\pagestyle{fancyplain} % Makes all pages in the document conform to the custom headers and footers
\fancyhead{} % No page header - if you want one, create it in the same way as the footers below
\fancyfoot[L]{} % Empty left footer
\fancyfoot[C]{} % Empty center footer
\fancyfoot[R]{\thepage} % Page numbering for right footer
\renewcommand{\headrulewidth}{0pt} % Remove header underlines
\renewcommand{\footrulewidth}{0pt} % Remove footer underlines
\setlength{\headheight}{13.6pt} % Customize the height of the header

\numberwithin{equation}{section} % Number equations within sections (i.e. 1.1, 1.2, 2.1, 2.2 instead of 1, 2, 3, 4)
\numberwithin{figure}{section} % Number figures within sections (i.e. 1.1, 1.2, 2.1, 2.2 instead of 1, 2, 3, 4)
\numberwithin{table}{section} % Number tables within sections (i.e. 1.1, 1.2, 2.1, 2.2 instead of 1, 2, 3, 4)

\setlength\parindent{0pt} % Removes all indentation from paragraphs - comment this line for an assignment with lots of text

%----------------------------------------------------------------------------------------
%	TITLE SECTION
%----------------------------------------------------------------------------------------

\newcommand{\horrule}[1]{\rule{\linewidth}{#1}} % Create horizontal rule command with 1 argument of height

\title{	
	\normalfont \normalsize 
	\textsc{Masarykova Univerzita, Fakulta Informatiky} \\ [25pt] % Your university, school and/or department name(s)
	\horrule{0.5pt} \\[0.4cm] % Thin top horizontal rule
	\huge Sada 2 \\ % The assignment title
	\horrule{2pt} \\[0.5cm] % Thick bottom horizontal rule
}

\author{Samuel Pastva (410286)} % Your name

\date{\normalsize\today} % Today's date or a custom date

\begin{document}
	
	\textbf{Modely distribuovaných systémů - základní pojmy a pricipy, synchronní a asynchronní komunikace. Synchronizace. Detekce ukončení.  Problém vzájemného vyloučení a problém uváznutí a jejich řešení.  Problém volby vedoucího prvku - vliv topologie a její znalosti/neznalosti na složitost řešení problému.}
	
	\section{Modely}
	
	\begin{itemize}
		\item Nezávislé procesy ktoré komunikujú správami po dopredu danej sieti. Problémy:
		
		\begin{itemize}
			\item Procesy sú nespoľahlivé: a) Neznámy execution time (pustí sa GC a čo teraz?) b) Proces môže zlyhať c) Proces môže podvádzať
			\item Sieť je nespoľahlivá: a) Správy môžu byť rôzne oneskorené b) Poradie správ sa môže meniť c) Správy môžu byť zdvojené/strácať sa d) Môžu vznikať falošné správy
		\end{itemize}
	
		\item Záleží na topológií siete — ring, star, úplný graf, mriežka, torus, strom, hypercube, unknown atd. — často sa na skutočnej topológií vytvorí virtuálna úplná topológia, lebo algoritmy sú potom jednoduššie.
		
		\item Komunikácia: Asynchrónna (send and forget), Synchrónna (mám potvrdenie o prijatí)
		
		\item Zložitosť: \emph{Komunikačná} (počet správ), časová, priestorová.
		
		\item Synchronizácia: No ordering, Causal ordering, Absolute ordering. Causal ordering: Lamport a Vector clock — happens before relácia. 
		
	\end{itemize}

	\section{Algoritmy}
	
	\subsection{Mutual Exclusion}
	
	\begin{itemize}
		\item Primitívny prístup: Na ringu koluje token.
		\item Lamport: Každý proces si drží frontu zoradenú podľa causal clock. Keď chcem ísť do kritickej sekcie, pošlem request všetkým a čakám kým mi všetci odpovedia. Ak som stále na vrchole fronty, môžem ísť do kritickej sekcie. Inak čakám na release. Na konci kritickej pošlem všetkým release, by vedeli že ma môžu vyhodiť z fronty. (lineárny)
		\item Raymond: Mám kostru na sieti a každý node má rodiča. Vždy si pýtam ktorým smerom je práve token a tým smerom posielam requesty. Každý uzol si pamätá requesty a keď skonči kritická sekcia, pošle token podľa request fronty. Celkovo len posielanie tokenu po strome miesto kruhu. (logaritmický)
		\item Maekawa: Mám kvóra veľkosti sqrt a vždy sa pýtam len môjho kvóra s tým že musí platiť že všetky kvóra majú pairwise neprázdny prienik.
	\end{itemize}
	
	\subsection{Termination Detection}
	
	\begin{itemize}
		\item Dijkstra-Scholten: Prvé requesty vybudujú strom v rámci ktorého ak niekto skončí so všetkým (mám counter na request/ack), tak notifikujem otca (ak sa stanem idle a znovu sa zobudím, môžem dostať nového otca, len ho nemôžem meniť počas aktivity). (lineárne)
		\item Safra: Každý proces si drží počet odoslané - prijaté správy a flag o tom, či od posledného posunutia tokenu pracoval. Posunutím tokenu sa flag nuluje. Na začiatku posielam čistý token s nulou. Procesy k tomuto pripočítavajú ich counter a ich flag. Ak mám na konci nulu a čistý flag v tokene aj u seba, končím. Ak nie, musím ísť odznova s čiernym/bielym tokenom podľa toho či som od minula pracoval. (counter potrebujem na eleiminaciu sprav v prechode, flag potrebujem kvoli predbiehaniu tokenu)
	\end{itemize}
	
	\subsection{Leader Election}
	
	\begin{itemize}
		\item Všetky procesy sú na začiatku rovnaké. Na konci má práve jeden flag leader.
		\item Pokiaľ procesy nemajú unikátne ID alebo nejaký zdroj náhody, election nemusí byť možná (symetrický systém).
		\item Stromy: Inicializácia z listov, vyhráva najmenšie ID (nemusím poznať koreň). Algoritmus: Zbieram ponuky (a držím si minimum) or mojich susedov. Pokiaľ mám len jedného suseda od ktorého som ponuku nedostal, pošlem mu ponuku ako môj súčasný stav (tu začínajú listy) a čakám kým mi sused nepošle odpoveď na ponuku (s nejakým ID). Aktualizujem minimum podľa tohoto výsledku a notifikujem všetkých ostatných. (lineárne správ)
		\item Ring (Chang-Roberts): Pošlem moje ID po kruhu. Forwardujem len IDčka ktoré sú menšie ako ja (a ak ich vidím viem že som prehral). Keď uvidím znova moja ID, som zvoelný a pošlem všetkým na kruhu víťaznú správu aby vedeli že už nemusia čakať. (kvadraticky sprav, priemer je nlogn) 
		\item Ring (Peterson): Algoritmus funguje po kolách pričom v každom kole sa moja skupina porovná so susedmi a pokiaľ nemá menšie ID, tak prehráva a pridáva sa k víťaznej skupine. Problém je že toto vyžaduje obojsmerný kruh. To sa obíde tak, že IDčka v každom kole rotujú o jeden proces ďalej. Takže ja ako proces si zistím ID predchodcu a potom moje ID (alebo ID predchodcu ak je menší) postuniem ďalšiemu procesu, ktorý ho porovná so sebou a osvojí si víťaza. (nlogn zložitosť)
		\item General: Voľba vlnou — V každom uzle sa iniciuje prehľadávanie siete, ale iba jeden uzol (leader) dostane ack že sieť je prehľadaná, lebo vlny sa budú navzájom rušiť až kým prežije len jedna (Obecne kvadratická zložitosť - N vĺn o veľkosti N)
		\item General (GHS): Voľba kostrou — Predpokladá váhy na hranách, aby mohol budovať min. kostru. Graf je rozdelený na fragmenty. Fragmenty sa v rámci seba dohodnú na minimálnej outgoing hrane. Pokiaľ za ňou leží menší/rovný fragment, tak sa k nemu pripojíme. Ak sme väčší, tak čakáme (iniciuje vždy ten menší). (NlogN + E)
		\item General (KKM): Voľba traversalom — algoritmus je zaujímavý, lebo ukazuje že zložitosť voľby je spojená so zložitosťou traversalu. Princíp: Fungujem podobne ako Peterson, po leveloch. Mám tri typy tokenov: annexing, chasing a waiting. Ak token stretne hocikoho s vyšším levelom, okamžite umiera. Ak stretne niekoho s rovnakým, tak sa pobijú a vznikne nový token ktorý je o level vyššie. Ak token prejde uzlom kde už bol token s lepším ID, tak sa z neho stane chsing token a pokúsi sa dobehnúť annektora a spojiť sa s ním. Ak annektor prejde uzlom kde bol naposledy token s horším ID, tak je waiting, lebo to znamená že ma niekto bude naháňať a ja ho chcem počkať. ((1 + logk) * f(N) - logaritmus krát zložitosť traversalu)
	\end{itemize}
\end{document}