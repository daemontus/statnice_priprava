%%%%%%%%%%%%%%%%%%%%%%%%%%%%%%%%%%%%%%%%%
% Short Sectioned Assignment
% LaTeX Template
% Version 1.0 (5/5/12)
%
% This template has been downloaded from:
% http://www.LaTeXTemplates.com
%
% Original author:
% Frits Wenneker (http://www.howtotex.com)
%
% License:
% CC BY-NC-SA 3.0 (http://creativecommons.org/licenses/by-nc-sa/3.0/)
%
%%%%%%%%%%%%%%%%%%%%%%%%%%%%%%%%%%%%%%%%%

%----------------------------------------------------------------------------------------
%	PACKAGES AND OTHER DOCUMENT CONFIGURATIONS
%----------------------------------------------------------------------------------------

\documentclass[paper=a4, fontsize=11pt]{scrartcl} % A4 paper and 11pt font size

%\usepackage[T1]{fontenc} % Use 8-bit encoding that has 256 glyphs
%\usepackage{fourier} % Use the Adobe Utopia font for the document - comment this line to return to the LaTeX default
\usepackage[utf8]{inputenc} 
\usepackage[czech]{babel} % English language/hyphenation
\usepackage{amsmath,amsfonts,amsthm} % Math packages
\usepackage{paracol}

\usepackage{lipsum} % Used for inserting dummy 'Lorem ipsum' text into the template

\usepackage{sectsty} % Allows customizing section commands
\allsectionsfont{\centering \normalfont\scshape} % Make all sections centered, the default font and small caps

\usepackage{fancyhdr} % Custom headers and footers
\pagestyle{fancyplain} % Makes all pages in the document conform to the custom headers and footers
\fancyhead{} % No page header - if you want one, create it in the same way as the footers below
\fancyfoot[L]{} % Empty left footer
\fancyfoot[C]{} % Empty center footer
\fancyfoot[R]{\thepage} % Page numbering for right footer
\renewcommand{\headrulewidth}{0pt} % Remove header underlines
\renewcommand{\footrulewidth}{0pt} % Remove footer underlines
\setlength{\headheight}{13.6pt} % Customize the height of the header

\numberwithin{equation}{section} % Number equations within sections (i.e. 1.1, 1.2, 2.1, 2.2 instead of 1, 2, 3, 4)
\numberwithin{figure}{section} % Number figures within sections (i.e. 1.1, 1.2, 2.1, 2.2 instead of 1, 2, 3, 4)
\numberwithin{table}{section} % Number tables within sections (i.e. 1.1, 1.2, 2.1, 2.2 instead of 1, 2, 3, 4)

\setlength\parindent{0pt} % Removes all indentation from paragraphs - comment this line for an assignment with lots of text

%----------------------------------------------------------------------------------------
%	TITLE SECTION
%----------------------------------------------------------------------------------------

\newcommand{\horrule}[1]{\rule{\linewidth}{#1}} % Create horizontal rule command with 1 argument of height

\title{	
	\normalfont \normalsize 
	\textsc{Masarykova Univerzita, Fakulta Informatiky} \\ [25pt] % Your university, school and/or department name(s)
	\horrule{0.5pt} \\[0.4cm] % Thin top horizontal rule
	\huge Sada 2 \\ % The assignment title
	\horrule{2pt} \\[0.5cm] % Thick bottom horizontal rule
}

\author{Samuel Pastva (410286)} % Your name

\date{\normalsize\today} % Today's date or a custom date

\begin{document}
	
	\textbf{Grafy a grafové algoritmy. Formalizace základních grafových pojmů, reprezentace grafů. Souvislost grafu, barevnost, rovinné grafy. Algoritmy (včetně složitosti a základní myšlenky důkazů korektnosti): prohledávání grafu do šířky a do hloubky, nejkratší vzdálenosti, kostry, toky v sítích.}
	
	\section{Základné grafové pojmy}
	
	\begin{itemize}
		\item Neorientovaný graf: $G = (V, E)$ kde $E \subseteq \mathcal{P}(V)$ a $x \in E \implies |x| = 2$. 

		\item Orientovaný graf: $G = (V, E)$ kde $E \subseteq V \times V$.

		\item Kružnica $C_n$ ($n$ hrán, $n$ vrcholov), Cesta $P_n$ ($n$ hrán, $n+1$ vrcholov), Úplný graf $K_n$ ($(\frac{n}{2})$ hrán, $n$ vrcholov), Úplný bipartitný graf $K_{m,n}$ ($m + n$ vrcholov, $m \cdot n$ hrán).
		
		\item Stupeň vrchola $d_G(v) = | \{\{u,v\} \mid \{u,v\} \in E(G) \} |$.
		
		\item Izomorfizmus grafov $G$ a $H$ je zobrazenie $f: V(G) \to V(H)$ t.ž. $\{u,v\} \in E(G) \iff \{f(u), f(v)\} \in E(H)$. Píšeme $G \simeq H$.
		
		\item Nezávislá množina: $X \subseteq V(G)$ t.ž. pre žiadne $u, v \in X$ nie je $\{u,v\} \in E$. Vrcholové pokrytie: Množina vrcholov taká, že všetky hrany končia/začínajú v množine. Dominujúca množina: každý vrchol má aspoň jednoho súseda v množine.
		
		\item Komponenta súvislosti: $X \subseteq G(V)$ t.ž. pre všetky $u, v \in X$ je $\{u, v\} \in E*$ (tranzitívny uzáver). V orientovaných grafoch silná/slabá súvislosť.
		
		\item Graf je súvislý ak je tvorený práve jednou komponentou (všetci vedia ísť všade).  
		
		\item Reprezečntácia grafov: Matica susednosti, Incidenčná matica, Zoznam hrán, Implicitne.
		
		\item Ofarbenie grafu je funkcia $c(v) : V \to \mathbb{N}$ t.ž. $\{u,v\} \in E$ tak $c(u) \not= c(v)$.
		
		\item Farebnosť grafu $\chi(G)$ je najmenšie $k$ také, že exisuje $c$ t.ž. $c(v) \leq k$ pre všetky $v$.
		
		\item \emph{Erdös:} Pre ľubovoľné $c,r$ existuje graf s farebnosťou aspoň $c$ a najmenšou kružnicou väčšou ako $r$.
		
		\item NP-úplné problémy: 3-farebnosť, Nezávislá množina (veľkosti aspoň $k$), Vrcholové pokrytie (veľkosti najviac $k$), Dominujúca množina (veľkosti najviac $k$), Hamiltonovský cyklus/cesta.
		
		\item Rovinný graf je t.ž. sa dá nakresliť do roviny bez pretínajúcich sa hrán. Stenami takéhoto zakreslenia sú topologicky súvislé oblasti (oddelené hranami). Existuje lineárny algoritmus na určenie rovinnosti grafu.
		
		\item Duálny graf: Steny nahradíme vrcholmi. Hrany vytvoríme podľa dotýkajúcich sa stien.
		
		\item \emph{Euler:} Nech $G$ je planárny graf ktorý má $f$ stien. Potom $|V| + f - |E| = 2$.
		
		\item \emph{Kuratowski:} Graf je rovinný práve vtedy, keď neobsahuje podrozdelenie $K_5$ alebo $K_{3,3}$ ako podgrafy.
		
	\end{itemize}
	
	\section{Grafové Algoritmy}
	
	\begin{itemize}
		
		\item BFS/DFS: vrchol — init, navštívený, spracovaný (hrany sú vo fronte), hrana — init, spracovaná. Preorder, Postorder.
		
		\item Najkratšie cesty (O - one-to-one, M - many-to-many): 
		
		\begin{itemize}
			\item (O) DAG a neohodnotený graf: BFS. Čas: $\mathcal{O}(V + E)$ Priestor: $\mathcal{O}(V + E)$.
			
			\item (O) Dijkstra: Zober najbližší nespracovaný a spracuj ho tak, že aktualizuješ vzdialenosť objavených vrcholov. Vždy si pamätaj odkiaľ sa naposledy znižovala vzdialenosť, aby si vedel zrekonštruovať cestu. Čas: $\mathcal{O}(E + V \cdot log(V))$ Priestor: $\mathcal{O}(V + E)$
			
			\item (M) Floyd-Warshall: V $k$-tej iterácií napočítam cesty ktoré používajú prvých $k$ vrcholov. Čas: $\mathcal{O}(V^3)$ Priestor: $\mathcal{O}(V^2)$
		\end{itemize}
	
		\item Toky v sieťach:
		
		\begin{itemize}
			\item Ford-Fulkerson: Vytvorím reziduálnu sieť a v nej hľadám zlepšujúcu cestu. 
			\item Edmonds-Karp: V reziduálnom grafe hľadám najkratšiu zlepšujúcu cestu.
			\item Ďalšie vylepšenia...
		\end{itemize}
	
		\item Minimálne kostry:
		
		\begin{itemize}
			\item Kostra: minimálny súvislý podgraf.
			\item Hladový: Hrany od najmenšej po najväčšiu. Ak som vytvoril cyklus, nepridávam hranu (je najväčšia na cykle).
			\item Jarník/Prim: Začnem z vrcholu a pridávam vždy najmenšiu border hranu.
			\item Borůvka: Pridávam globálne najmenšiu hranu do nepokrytého vrchola.
		\end{itemize}
	
		\item \textbf{TODO dokazy a zlozitost}
	\end{itemize}
\end{document}