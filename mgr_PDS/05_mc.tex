%%%%%%%%%%%%%%%%%%%%%%%%%%%%%%%%%%%%%%%%%
% Short Sectioned Assignment
% LaTeX Template
% Version 1.0 (5/5/12)
%
% This template has been downloaded from:
% http://www.LaTeXTemplates.com
%
% Original author:
% Frits Wenneker (http://www.howtotex.com)
%
% License:
% CC BY-NC-SA 3.0 (http://creativecommons.org/licenses/by-nc-sa/3.0/)
%
%%%%%%%%%%%%%%%%%%%%%%%%%%%%%%%%%%%%%%%%%

%----------------------------------------------------------------------------------------
%	PACKAGES AND OTHER DOCUMENT CONFIGURATIONS
%----------------------------------------------------------------------------------------

\documentclass[paper=a4, fontsize=11pt]{scrartcl} % A4 paper and 11pt font size

%\usepackage[T1]{fontenc} % Use 8-bit encoding that has 256 glyphs
%\usepackage{fourier} % Use the Adobe Utopia font for the document - comment this line to return to the LaTeX default
\usepackage[utf8]{inputenc} 
\usepackage[czech]{babel} % English language/hyphenation
\usepackage{amsmath,amsfonts,amsthm} % Math packages
\usepackage{paracol}

\usepackage{lipsum} % Used for inserting dummy 'Lorem ipsum' text into the template

\usepackage{sectsty} % Allows customizing section commands
\allsectionsfont{\centering \normalfont\scshape} % Make all sections centered, the default font and small caps

\usepackage{fancyhdr} % Custom headers and footers
\pagestyle{fancyplain} % Makes all pages in the document conform to the custom headers and footers
\fancyhead{} % No page header - if you want one, create it in the same way as the footers below
\fancyfoot[L]{} % Empty left footer
\fancyfoot[C]{} % Empty center footer
\fancyfoot[R]{\thepage} % Page numbering for right footer
\renewcommand{\headrulewidth}{0pt} % Remove header underlines
\renewcommand{\footrulewidth}{0pt} % Remove footer underlines
\setlength{\headheight}{13.6pt} % Customize the height of the header

\numberwithin{equation}{section} % Number equations within sections (i.e. 1.1, 1.2, 2.1, 2.2 instead of 1, 2, 3, 4)
\numberwithin{figure}{section} % Number figures within sections (i.e. 1.1, 1.2, 2.1, 2.2 instead of 1, 2, 3, 4)
\numberwithin{table}{section} % Number tables within sections (i.e. 1.1, 1.2, 2.1, 2.2 instead of 1, 2, 3, 4)

\setlength\parindent{0pt} % Removes all indentation from paragraphs - comment this line for an assignment with lots of text

%----------------------------------------------------------------------------------------
%	TITLE SECTION
%----------------------------------------------------------------------------------------

\newcommand{\horrule}[1]{\rule{\linewidth}{#1}} % Create horizontal rule command with 1 argument of height

\title{	
	\normalfont \normalsize 
	\textsc{Masarykova Univerzita, Fakulta Informatiky} \\ [25pt] % Your university, school and/or department name(s)
	\horrule{0.5pt} \\[0.4cm] % Thin top horizontal rule
	\huge Sada 2 \\ % The assignment title
	\horrule{2pt} \\[0.5cm] % Thick bottom horizontal rule
}

\author{Samuel Pastva (410286)} % Your name

\date{\normalsize\today} % Today's date or a custom date

\begin{document}
	
	\textbf{Metoda ověřování modelu (MC) pro konečně stavové systémy a lineární temporální logiku. Princip překladu formulí LTL na automaty nad nekonečnými slovy.  Základní symbolické a explicitní algoritmy pro MC a jejich teoretická složitost. }
	
	\section{LTL model checking}
	
	\begin{itemize}
		\item TODO: Uplnost Hoareho logiky? Unknown?
		
		\item Nedeterminizmus typicky vzniká ako dôsledok paralelizmu.
		
		\item Formálny model systému: Kripkeho štruktúra — stavy $S$, prechodová relácia $T \subseteq S \times S$ (väčšinou sa predpokladá totálna, ak nie, opakujem deadlock stav), labelling $I \subseteq S \times \mathcal{P}(AP)$, iniciálny stav $s_0$. Prípadne môžem mať ešte prechody označené akciami — Kripkeho prechodový systém — (typicky jedna akcia je vykonávaná viacerými prechodmi, podľa toho v akom stave sú napr. ostatné procesy). 
		
		\item Buchi automaton — $\mathcal{B} = (\Sigma, S, s, \delta, F)$, $\Sigma$ je abeceda, $S$ sú stavy, $s$ je inicálny stav, $\delta \subseteq S \times \Sigma \times S$ je prechodová relácia a $F$ sú akceptujúce stavy. Beh akceptuje ak je niečo z $F$ nekonečne krát navštívené. 
		
		\item LTL formula je preložiteľná na Buchi automat. Kripkeho štruktúra je preložiteľná na Buchi automat (abeceda je powerset propozícií, akceptujem všetkými stavmi - všetky behy akceptujú).
		
		\item Synchronna kompozícia týchto dvoch automatov je triviálna, keďže jeden automat akceptuje všetkými stavmi, teda ako akceptujúce stavy mi stačí kartézsky súčin (inak by som musel riešiť preskakovanie z tieru do tieru aby som zabezpečil že sa budú opakovať stavy z oboch automatov)
		
		\item Teda $M \models \varphi \iff L(M) \subseteq L(\varphi) \iff L(M) \cap co-L(\varphi) = \emptyset \iff L(M) \cap L(\neg\varphi) = \emptyset \iff L(M \times \neg\varphi) = \emptyset$ ($\varphi$ a $M$ preťažujem, takže je to buď pôvodný systém/formula alebo už konkrétny Buchi automaton)
		
		\item Buchi reprezentuje neprázdny jazyk práve vtedy keď existuje dosiahnuteľný akceptujúci cyklus (laso).
		
		\item Algoritmus na hľadanie akceptujúceho cyklu: Nested DFS - zložitosť je lineárna k veľkosti systému, lebo keď sa vynorím z "nested" časti, určite nevedie žiadna nepreskúmaná hrana von, a teda tam nemôže byť cyklus do ktorého by som vedel vstúpiť z iného miesta.
		
		\item NestedDFS: Dve prehľadávania — Prvé vypočíta post–order akceptujúcich stavov z daného inicálneho stavu. Druhé potom kontroluje prítomnosť cyklu pre jednotlivé akceptujúce stavy, ale kvôli nested argumentu nemusí ísť do už preskúmaných častí grafu. — nepotrebujem MC, stačí reachabilita
		
		\item Terminálny Buchi automat - všetky akceptujúce cykly sú selfloopy - safety vlastnosti.
		\item Slabý Buchi automat - všetky komponenty obsahujú buď len akceptujúce cykly, alebo len neakceptujúce (nemám zmiešané komponenty) - weak vlastnosti (typicky nejake $G(x \implies F y)$) — nepotrebujem NestedDFS, stačí DFS.
		
	\end{itemize}

	\section{LTL to BA}
	
	\begin{itemize}
		\item Formula v normálnom tvare — negácia na literáloch a všetko je prevedené na X, U a R.
		\item Použije sa zobecnený automat — mám viacero množín pričom beh musí nekonečne krát prejsť každou z nich.
		\item Preklad zobecneného automatu na štandardný: Zreplikujem stavový priestor pre každú množinu + jeden extra nultý tier. Do ďalšieho tieru môžem postúpiť vždy len cez akceptujúci stav (teda do tieru $k$ sa dostanem ak som v tiere $k-1$ a VSTUPUJEM do akceptujúceho stavu množiny $F_k$). Akceptujúce stavy sú všetky stavy posledného (alebo vlastne hociktorého) tieru. Teda pokiaľ mám nekonečný akceptujúci beh, tak to znamená že prejde akceptujúcim stavom každého z tierov.
		
		\item Konštrukcia grafu k formuli: Rekurízvne po štruktúre formule. Pre každý stav grafu si pamätám čo tam už mám $Now$, čo tam chcem $New$ a čo má platiť v ďalšom stave $Next$. Začínam s jedným stavom ktorý má v $New$ celú formulu a postupne ju rozbaľujem.
		
		\item Ak v $New$ propozíciu, pozriem sa či je splniteľná s aktuálnym $Now$, ak nie, mažem tento uzol, ak ano, pridávam následníka ako presnú kópiu, iba s pridanou propozíciou. Ak mám and, vytváram nového následníka v ktorom do $New$ dám obidve podformule, ak mám or, pridávama dvoch následníkov kde každý má v $New$ jednu podformulu. Ak mám until, vytvorím dvoch následníkov — jeden má v $New$ path formulu a v $Next$ má zase until a druhý má v $New$ reach formulu. Pre release mám jedného čo má "released" formulu v new a release v next a druhého čo má "released" aj "releaser" v new a v next nemá nič nové.
		\item Vo všetkých následníkoch samozrejme platí expandovaná formula v $Now$. Plus pokiaľ už nie je čo expandovať, tak skontrolujem či som náhodou nevyrobil niečo čo už mám, a ak ano, tak len incoming hrany do duplicitného stavu presmerujem tam. Ak som skutočne vyrobil nový stav, tak ho pridám a vytvorím nový stav, ktorý bude mať v $New$ veci z môjho aktuálneho $Next$.
	\end{itemize}
	
\end{document}