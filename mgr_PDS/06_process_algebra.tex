%%%%%%%%%%%%%%%%%%%%%%%%%%%%%%%%%%%%%%%%%
% Short Sectioned Assignment
% LaTeX Template
% Version 1.0 (5/5/12)
%
% This template has been downloaded from:
% http://www.LaTeXTemplates.com
%
% Original author:
% Frits Wenneker (http://www.howtotex.com)
%
% License:
% CC BY-NC-SA 3.0 (http://creativecommons.org/licenses/by-nc-sa/3.0/)
%
%%%%%%%%%%%%%%%%%%%%%%%%%%%%%%%%%%%%%%%%%

%----------------------------------------------------------------------------------------
%	PACKAGES AND OTHER DOCUMENT CONFIGURATIONS
%----------------------------------------------------------------------------------------

\documentclass[paper=a4, fontsize=11pt]{scrartcl} % A4 paper and 11pt font size

%\usepackage[T1]{fontenc} % Use 8-bit encoding that has 256 glyphs
%\usepackage{fourier} % Use the Adobe Utopia font for the document - comment this line to return to the LaTeX default
\usepackage[utf8]{inputenc} 
\usepackage[czech]{babel} % English language/hyphenation
\usepackage{amsmath,amsfonts,amsthm} % Math packages
\usepackage{paracol}

\usepackage{lipsum} % Used for inserting dummy 'Lorem ipsum' text into the template

\usepackage{sectsty} % Allows customizing section commands
\allsectionsfont{\centering \normalfont\scshape} % Make all sections centered, the default font and small caps

\usepackage{fancyhdr} % Custom headers and footers
\pagestyle{fancyplain} % Makes all pages in the document conform to the custom headers and footers
\fancyhead{} % No page header - if you want one, create it in the same way as the footers below
\fancyfoot[L]{} % Empty left footer
\fancyfoot[C]{} % Empty center footer
\fancyfoot[R]{\thepage} % Page numbering for right footer
\renewcommand{\headrulewidth}{0pt} % Remove header underlines
\renewcommand{\footrulewidth}{0pt} % Remove footer underlines
\setlength{\headheight}{13.6pt} % Customize the height of the header

\numberwithin{equation}{section} % Number equations within sections (i.e. 1.1, 1.2, 2.1, 2.2 instead of 1, 2, 3, 4)
\numberwithin{figure}{section} % Number figures within sections (i.e. 1.1, 1.2, 2.1, 2.2 instead of 1, 2, 3, 4)
\numberwithin{table}{section} % Number tables within sections (i.e. 1.1, 1.2, 2.1, 2.2 instead of 1, 2, 3, 4)

\setlength\parindent{0pt} % Removes all indentation from paragraphs - comment this line for an assignment with lots of text

%----------------------------------------------------------------------------------------
%	TITLE SECTION
%----------------------------------------------------------------------------------------

\newcommand{\horrule}[1]{\rule{\linewidth}{#1}} % Create horizontal rule command with 1 argument of height

\title{	
	\normalfont \normalsize 
	\textsc{Masarykova Univerzita, Fakulta Informatiky} \\ [25pt] % Your university, school and/or department name(s)
	\horrule{0.5pt} \\[0.4cm] % Thin top horizontal rule
	\huge Sada 2 \\ % The assignment title
	\horrule{2pt} \\[0.5cm] % Thick bottom horizontal rule
}

\author{Samuel Pastva (410286)} % Your name

\date{\normalsize\today} % Today's date or a custom date

\begin{document}
	
	\textbf{Formalismy pro modelování nekonečně stavových systémů (Petriho sítě, procesové přepisovací systémy, automaty, procesové kalkuly) a algebry procesů, porovnání jejich vyjadřovací síly vzhledem k bisimulaci. Vybrané rozhodnutelné problémy z oblasti verifikace těchto systémů. }
	
	\section{Process Rewrite Systems}
	
	\begin{itemize}
		\item Bavime sa o labeled transition systemoch...
		\item Slabá ekvivalencia: Trace equivalence.
		\item Silná ekvivalencia: Bisimulácia. Relácia $R$ je bisimulácia pokiaľ $(s, t) \in R$ a $(s, a, s') \in \Delta$, tak existuje $t'$ t.ž. $(t, a, s') \in \Delta$ a $(s', t') \in R$ a to isté opačným smerom. Chceme maximálnu bisimuláciu.
		
		\item Process Rewrite Systems: Množina akcií $a,b,c ...$, množina procesových konštant $X, Y, Z, ...$ Term — $t ::= \epsilon \mid X \mid t_1 \cdot t_2 \mid t_1 || t_2$ a množina transformácií (term, action, term).
		
		\item Syntaktická ekvivalencia na PRS: Axiómy: Nezáleží na uzávorkovaní sekvešnej ani paralelnej kompozície (len kombinácia je problém). Môžem ľubovoľne zamieňať poradie paralelných termov. Môžem pridávať/odoberať $\epsilon$ kam len chcem. Odvodzovacie pravidlá: môžem transformovať prvý term v sekvenčnej/paralelnej kompozícií.
		
		\item LTS potom z tohto vznikne tak, že stavy sú všetky možné termy a prechody pridávam podľa možných transformácií (s tým že môžem používať syntaktickú ekvivalenciu, takže v paralelnom procese môžem aplikovať na ľubovoľný, ale v sekvenčnom len na prvý proces).
		
		\item Hierarchia termov:
		
		\begin{itemize}
			\item \textbf{1} Obsahuje len konštanty
			\item \textbf{S} Obsahuje len sekvenčnú kompozíciu a konštanty.
			\item \textbf{P} Obsahuje len paralelnú kompozíciu a konštanty.
			\item \textbf{G} Obsahuje hocičo.
		\end{itemize}
	
		\item Hierarchia procesov (dve obmedzenia — čo môže byť na ľavej a na pravej strane pravidla (plus nemá význam mať viac general veci na ľavo)):
		
		\begin{itemize}
			\item (1,1) sú presne končené automaty, regulárne jazyky a spol.
			\item (1,P) sú BPP — Basic Parallel Processes
			\item (1, S) sú BPA — Basic Process Algebra
			\item (1, G) je Process Algebra
			\item (P, P) sú Petriho siete
			\item (S, S) sú Pushdown automaty
			\item (S,G), (P,G) a (G,G) nemajú meno.						
		\end{itemize}
		
		\item Hierarchia je striktná vzhľadom na bisimuláciu.
		
	\end{itemize}

	\section{Verifikačné problémy}

	\begin{itemize}
		\item Reachability pre Petri Nets je rozhodnuteľná. 
		\item LTL pre Pushdown je rozhodnuteľné (repeating heads).
		\item Equivalence checking je rozhodnuteľný pre (1,1), (1, P), (1, S) a (S, S).
		\item TODO?
	\end{itemize}
\end{document}