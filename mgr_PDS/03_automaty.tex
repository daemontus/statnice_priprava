%%%%%%%%%%%%%%%%%%%%%%%%%%%%%%%%%%%%%%%%%
% Short Sectioned Assignment
% LaTeX Template
% Version 1.0 (5/5/12)
%
% This template has been downloaded from:
% http://www.LaTeXTemplates.com
%
% Original author:
% Frits Wenneker (http://www.howtotex.com)
%
% License:
% CC BY-NC-SA 3.0 (http://creativecommons.org/licenses/by-nc-sa/3.0/)
%
%%%%%%%%%%%%%%%%%%%%%%%%%%%%%%%%%%%%%%%%%

%----------------------------------------------------------------------------------------
%	PACKAGES AND OTHER DOCUMENT CONFIGURATIONS
%----------------------------------------------------------------------------------------

\documentclass[paper=a4, fontsize=11pt]{scrartcl} % A4 paper and 11pt font size

%\usepackage[T1]{fontenc} % Use 8-bit encoding that has 256 glyphs
%\usepackage{fourier} % Use the Adobe Utopia font for the document - comment this line to return to the LaTeX default
\usepackage[utf8]{inputenc} 
\usepackage[czech]{babel} % English language/hyphenation
\usepackage{amsmath,amsfonts,amsthm} % Math packages
\usepackage{paracol}

\usepackage{lipsum} % Used for inserting dummy 'Lorem ipsum' text into the template

\usepackage{sectsty} % Allows customizing section commands
\allsectionsfont{\centering \normalfont\scshape} % Make all sections centered, the default font and small caps

\usepackage{fancyhdr} % Custom headers and footers
\pagestyle{fancyplain} % Makes all pages in the document conform to the custom headers and footers
\fancyhead{} % No page header - if you want one, create it in the same way as the footers below
\fancyfoot[L]{} % Empty left footer
\fancyfoot[C]{} % Empty center footer
\fancyfoot[R]{\thepage} % Page numbering for right footer
\renewcommand{\headrulewidth}{0pt} % Remove header underlines
\renewcommand{\footrulewidth}{0pt} % Remove footer underlines
\setlength{\headheight}{13.6pt} % Customize the height of the header

\numberwithin{equation}{section} % Number equations within sections (i.e. 1.1, 1.2, 2.1, 2.2 instead of 1, 2, 3, 4)
\numberwithin{figure}{section} % Number figures within sections (i.e. 1.1, 1.2, 2.1, 2.2 instead of 1, 2, 3, 4)
\numberwithin{table}{section} % Number tables within sections (i.e. 1.1, 1.2, 2.1, 2.2 instead of 1, 2, 3, 4)

\setlength\parindent{0pt} % Removes all indentation from paragraphs - comment this line for an assignment with lots of text

%----------------------------------------------------------------------------------------
%	TITLE SECTION
%----------------------------------------------------------------------------------------

\newcommand{\horrule}[1]{\rule{\linewidth}{#1}} % Create horizontal rule command with 1 argument of height

\title{	
	\normalfont \normalsize 
	\textsc{Masarykova Univerzita, Fakulta Informatiky} \\ [25pt] % Your university, school and/or department name(s)
	\horrule{0.5pt} \\[0.4cm] % Thin top horizontal rule
	\huge Sada 2 \\ % The assignment title
	\horrule{2pt} \\[0.5cm] % Thick bottom horizontal rule
}

\author{Samuel Pastva (410286)} % Your name

\date{\normalsize\today} % Today's date or a custom date

\begin{document}
	
	\textbf{Konečné automaty (FA) a logiky nad slovy. Logika 1.řádu (FOL) a monadická logika 2.řádu (MSOL): syntax a sémantika FOL a MSOL, principy převoditelnosti mezi FA a formulemi MSOL. Automaty nad nekonečnými slovy a omega-regulární jazyky.}
	
	\section{FOL a MSOL nad slovami}
	
	\begin{itemize}
		\item First order logic: Štandardná FOL kde kvantifikujem cez pozície v slove a mám k dispozícií predikáty $x < y$ a $Q_a(x)$ - teda pozícia $x$ je pred pozíciou $y$ a na pozícií x sa nachádza znak $a$.
		\item Pokiaľ má formula voľné premenné, potrebujem ešte k tomu interpretáciu.
		\item Pozn.: Použitím $\exists$ vynucujem že slovo je neprázdne.
		\item Môžem si dodefinovať predikáty ako first/last (neexsituje $y$ menší/väčší ako $x$), prípadne next pozíciu (neexsituje $z$ väčšie ako $x$ a menšie ako $y$).
		\item FOL nemá ani vyjadrovaciu silu regulárnych jazykov (jazyk slov párnej dĺžky nie je vyjadriteľný). 
		\item Monadic second order logic: Rozšírime FOL o kvantifikáciu nad množinami pozícií a predikát $x \in X$.
		\item Teraz môžem vyjadriť napr. predikát Even (x je v množine iff je druhé alebo existuje pozícia v množine ktorá je o dve predomnou) a jazyk párne dlhých slov (existuje množina pozícií ktoré sú párne a posledná pozícia slova je v tejto množine).
		
		\item MSOL je ekvivalentná regulárnym jazykom. Dôkaz prekladom:
		
		\begin{itemize}
			\item FA $(\Sigma, Q, \delta, q_0, F)$ to MSOL (okrem prázdneho slova, to sa musí pridať do formule ešte explicitne):
			\item 1) Pozície $i$ v slove sú rozdelené do $n$ množín ($n$ je počet stavov), podľa toho, či sa automat nachádza v danom stave po prečítaní prvých $i$ znakov nejakého slova. Pozn.: Tu už uvažujeme "pre fixné slovo", keďže to je to o čom sa formula vyjadruje.
			\item 2) Skonštruujeme predikát $visits(X_0....X_n)$ ktorý toto vyjadruje. Konkrétne chceme povedať že: a) Existuje $x$ t.ž. $x$ je prvá pozícia a ak $Q_a(x)$ a súčasne $q_i \in (q_0, a)$, potom $x \in X_i$ (inicializácia). b) Množiny pozícií sú dizjunktné a každá pozícia patrí do nejakej množiny c) Pozície zachovávajú prechodovú funkciu, teda ak mám $q_i \in (q_j, a)$, mám $x \in X_j$ a $Q_a(x)$, tak $y = x + 1$ patrí do $X_i$.
			\item 3) S takýmto predikátom potom môžem povedať že existujú tieto množiny a súčasne posledná pozícia leží v množine odpovedajúcej akceptujúcemu stavu.
			
			\item MSOL to FA:
			\item Abeceda automatu je pôvodná abeceda, plus jeden bit pre každú voľnú premennú formule, ktorý vyjadruje či zrovna som, alebo nie som na tejto pozícií (res. na pozícií v tejto množine). Automat potom budujem induktívne:
			\item Pre $Q_a(x)$ — mám jednu voľnú premennú, dva stavy medzi kyotými sa môžem presunúť pod $[a, 1]$, inak cyklím.
			\item $x < y$ - mám dve premenné a tri stavy nad ktorými cyklím. Prvý presun robím pri $[?, 1, 0]$, druhý pri $[?, 0, 1]$ ($?$ znamená že mi je jedno aké písmeno čítam).
			\item $x \in X$ - mám dve premenné a dva stavy, presúvam sa pod $[?, 1, 1]$.
			\item $\varphi = \not \psi$ — nemôžem urobiť hlúpy komplement, lebo by som mohol akceptovať slová ktoré nie sú platnými kódmi (napr. premenná $x$ by mohla mať viac hodnôť). Môžem ale urobiť komplement a sprienikovať to s automatom pre povolené slová (ktorý viem vyrobiť ak viem aká je štruktúra mojich premenných).
			\item $\varphi = \varphi \lor \varphi$ — musím vhodne rozšíriť abecedy oboch automatov, aby obsahovali rovnaké premenné (tj. pridávam nové pozície, pričom ak mám na prechode $[a, 0]$ tak zamením za $[a, 0, 1]$ aj $[a, 0, 0]$). Potom môžem urobiť obyčajné zjednotenie.
			\item $\varphi = \exists x$ (alebo $X$, je to jedno) — odstránim danú pozíciu zo symbolov (fakt, nič viac).
			\item Veľkosť výsledného automatu môže byť až v ráde "veži exponentov" pričom výška veže je úmerná počtu negácií.
		\end{itemize}
	
	\end{itemize}

	\section{Omega automaty}
	
	\begin{itemize}
		\item Labelled Transition System: $(S, \Sigma, \Delta, S_{in})$ — stavy, abeceda, prechodová relácia, inicálne stavy.
		\item By default, všetko je nedeterministické!
		\item Buchi automaton: $(\Sigma, Q, \delta, s_0, F)$ - akceptuje ak existuje beh, ktorý navštívi $F$ nekonečne krát.
		\item Uzavretosť Buchi: Zjednotenie (trivialne ak dovolim viac inicialnych stavov, inak pridam novy init), Prienik (tiered konštrukcia), Regulárny.Buchi (proste zreťazím), Regulárny$^\omega$ (z akceptujúcych stavov sa zacyklím do iniciálneho)
		\item $\omega$-regulárne jazyky sú tvaru $A.B^\omega$ kde $A$ a $B$ sú regulárne jazyky a sú rozpoznávané práve Buchiho automatmi.
		\item Neprázdnosť omega jazyka je rozhodnuteľná (NLOG úplná ako reachabilita). Neuniverzalita je PSPACE.
		\item Buchi su uzavrete aj na komplement, ale dokaz je TODO.
		\item Zobecnený Buchi - prienik automatov s rovnakými stavmi a prechodmi, len inými accept množinami.
		
		\item Muller — podmienka: množiny stavov, slovo akceptuje ak inf je presne jedna z množín. Je silou ekvivalentná s Buchi, ale na vyjadrenie všetkých jazykov mi stačia deterministické automaty.
		\item Rabin — podmienka: mám dvojice množín stavov, slovo akceptuje ak inf má prienik s prvou množinou a nemá prienik s druhou. Tiež sú ekvivalentné s Buchi.
		\item Street — podmienka: tiež dvojice množín, slovo akceptuje ak inf má prienik s prvou, tak má prienik aj s druhou.
	\end{itemize}

\end{document}